\documentclass[a4paper,11pt]{article}

\usepackage{geometry}
\usepackage{url}
\usepackage{hyperref}
\usepackage{fancyhdr}
\usepackage{titlesec}
\usepackage{fontspec}
\usepackage{etoolbox}
\usepackage[backend=biber,
           bibstyle=numeric,sorting=ydnt,sortcites=true,natbib=true,defernumbers=true,
           maxbibnames=99,giveninits=true,uniquename=false]{biblatex}
\addbibresource{research.bib}
\addbibresource{poster.bib}
\addbibresource{ongoing.bib}

%\patchcmd{\thebibliography}{\section*{\refname}}{}{}{}

% Document layout
\geometry{tmargin=1.5cm, bmargin=1.5cm,lmargin=1.5cm, rmargin=1.5cm}
\setlength\parindent{0in}

%Fonts
\usepackage{xunicode}
\usepackage{xltxtra}
\defaultfontfeatures{Mapping=tex-text}
\usepackage[rm,light]{roboto}


\newcommand{\fullhrulefill}{%
  \hspace*{-\sectionwidth}\hrulefill%
  }

\title{\LARGE\bfseries Mutaz M. \textbf{Jaber}}
\author{jaber038@umn.edu}
\date{}

\begin{document}
\rule{\textwidth}{1pt}
~\\[0.5cm]
%\maketitle
{\centering\textbf{\LARGE Mutaz M. Jaber}\\[0.1cm] \texttt{jaber038@umn.edu} \\[-0.2cm]}
~\\[0.1cm]
1245 Ray Place \hfill Personal website:\href{https://mutaz94.github.io/}{https://mutaz94.github.io}\\
Falcon Heights MN 55108, United States\hfill Google scholar:\href{https://scholar.google.com/citations?user=uugc-IgAAAAJ&hl=en&authuser=1}{\texttt{Mutaz M. Jaber}}\\
Phone: +1 651-706-5202\hfill GitHub:\href{https://github.com/Mutaz94}{\texttt{mutaz94}}\\
\rule{\textwidth}{1pt}

\textbf{Research interest - Scientific skills}\\
Pharmacometrics, Clinical pharmacology, Nonlinear mixed-effect approach, Absorption models, computational statistics, estimation algorithms, Circadian rhythm models, Congenital adrenal hyperplasia, Machine learning\\

\textbf{Education}\\[-0.3cm]

Experimental and Clinical Pharmacology, PhD - University of Minnesota\hfill 2020-current\\
Emphasis: \texttt{Pharmacometrics}\\
Minor: \texttt{Statistics}\\
Advisor: Richard C. Brundage, PharmD, PhD\\
Co-advisor: Mahmoud Al-kofahi, DDS, PhD\\
Overall GPA: 3.95
Thesis: \emph{Evaluation of Pharmacostatistical Model Components in a Nonlinear Mixed-effect Approach}   
~\\[0.5cm]
Doctor of Pharmacy, PharmD - Jordan University of Science and Technology\hfill 2012-2018\\
Department of Clinical Pharmacy\\
College of Pharmacy\\[1cm]

\textbf{Research experiance}\\[-0.3cm]


\textbf{Graduate research assistant} - University of Minnesota \hfill 2020 - current \\
Advisor: Richard C. Brundage, PharmD, PhD \\
Project(s):
\begin{itemize}
\item Methodology research - Uncertainty in NLME approach, Machine learning
\item Clinical research - Hydrocortisone Pump for congenital adrenal hyperplasia patients
\end{itemize}


\textbf{Postdoctoral Associate} - University of Minnesota\hfill 2019-2020\\
Supervisor: Richard C. Brundage, PharmD, PhD\\
Project(s):\\
\begin{itemize}
\item Hydrocortisone pharmacokinetics/pharmacodynamics
\item Mathematical modeling of circadian rhythms
\item Tools development: HydroC-Precision
\item Absorption models
\end{itemize}

\textbf{Postgraduate researcher} - Jordan University of Science and Technology \hfill 2018 - 2019\\
Supervisor(s): Mera Ababneh, PharmD, PhD; Sayer AlAzzam, MSc, PharmD \\
Project(s):
\begin{itemize}
\item Antimicrobial stewardship
\item Pharmacostatistics models in pharmacoepidemiological research
\item Seasonal Influenza vaccination in Jordan
\end{itemize}

\textbf{Student research assistant} - Jordan University of Science and Technology \hfill 2017 - 2018\\
Project(s):
\begin{itemize}
\item Autism, and ADHD pediatric research
\item Protien chemical analysis
\end{itemize}

\textbf{Pharmacometrics summer training} \\
Pharmacometrics intern - Metrum Research Group \hfill June 2022 - Present \\
Mentor(s): Curtis Johnston, Kyle T. Baron \\[1cm]
Pharmacometrics GDD intern - Novartis \hfill June 2021 - Aug 2021\\
Mentor(s): Matthew Fidler \\
Project: Simulation uncertainty in NLME approach \\
\textbf{Clinical experiance}\\

Pharmacy intern - Drug information center (KAUH) \hfill May,2018-July,2018 \\

Clinical pharmacy rotation - KAUH \hfill Sep, 2017 - May, 2018 \\

Pharmacy intern - KHCC \hfill Jan 2017 - Mar 2017 \\

\textbf{Teaching experince} \\
\begin{itemize}
\item Graduate pharmacometrics summer series (2020, 2021): Population approach and NONMEM.
~\\
Teaching introduction to population pharmacokinetic focusing on nonlinear mixed-effect approach. This course includes both theoretical and hands on session. The main objective of this course is to introduce the concept and teach student using NONMEM.
~\\
No. of students: 15
\end{itemize}
~\\
\textbf{Programming skills}
~\\
Python, Julia, R, shell, FORTRAN 90+, C++, HTML5 and \TeX/\LaTeX\\
~\\
\textbf{Awards} \\

\begin{itemize}
\item ACCP/ISoP Student abstract award \hfill ACCP annual meeting 2020
\end{itemize}

\textbf{Scientific software development} \\
\begin{itemize}
\item HydroC-Precision: Integrated hydrocortisone PKPD platform for children with CAH.
\item NCA-ADAPT: Open-source program to conduct a pharmacokinetic non-compartmental analysis.
\item ABS-NN: Program to classify absorption shapes using neural networks.
\end{itemize}

\textbf{Ongoing projects/collaborations}\\
1- Subcutaneous Hydrocortisone Children With Congenital Adrenal Hyperplasia (NCT03718234) (2019-Current)
~\\
Role: PhD student: Develop PK/PD models to describe the dynamic of HPA axis (exogenous and endogenous components)
~\\
2- Office of Orphan Products Development of the US Food and Drug Administration award number R01FDR0006100. (2019-2020)
~\\
Role: Postdoctoral associate: Develop Oral Hydrocortisone PK/PD model
~\\
3- Describing methadone pharmacokinetics in opioid dependenent population (2021-current)
~\\
Role: Develop PK/PD models
~\\
4- Simulating clinical trials with uncertainty - Novartis Pharmacometrics
~\\
Role: Investigating the use of different sampling distribution for simulation.
~\\
\nocite{*}
\printbibliography[title=Publications, keyword=papers]

\printbibliography[title=In preparation/In Press, keyword=ongoing]

\printbibliography[title=Posters \& abstracts,keyword=poster]
\newpage
{\centering\textbf{Presentations \& talks}}
\begin{itemize}
    \item Reference Growth Charts in Children with Congenital Adrenal Hyperplasia \hfill ESPE 2022
    \item Case Series: Anastrozole Monotherapy for Non-Classic Congenital Adrenal Hyperplasia \hfill ESPE 2022
    \item Investigating the Contribution of Residual Unexplained Variability Components in a NLME approach \hfill  PAGE 2021 \& ECP 2021
    \item Simulation with Uncertainty in NLME approach \hfill Novartis 2021
    \item Application of Deep Learning in Pharmacokinetic Analysis \hfill ECP 2020
    \item HydroC-Precision: An Integrated Hydrocortisone Dosing and Biomarker Platform for Treating Children with Congenital Adrenal Hyperplasia \hfill ACCP 2020
\end{itemize}
\textbf{Professional service}
\begin{itemize}
\item Minnesota Pharmacometrics Summer School 2020: Planning Committee Member
\item International Society of Pharmacometrics: Student Committee Member
\item Doctor of Pharmacy Toward Optimum Patient Care Campaign
\item Cooperative Group of Medical Profession (CGMP)
\item  Jordan University of Science and Technology Student Union 2015-2016: Class representative to ’12 batch
\end{itemize}
\textbf{Professional society}
\begin{itemize}
    \item American Society of Clinical Pharmacology
    \item International Society of Pharmacometrics
    \item Society for Industrial and Applied Mathematics
    \item American Statistical Association - Pharmacometrics
\end{itemize}
\end{document}
